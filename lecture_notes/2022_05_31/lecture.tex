\documentclass[11pt]{article}
\usepackage{ {util/personalmacros} }
\usepackage{tikz-cd}
%\usetikzlibrary{arrows,calc, angles ,patterns}



\begin{document}
\maketitle
\begin{remark}
Nachdem proj. Unterraum und Schnitt und Quadriken invariant sind unter projektiven Transformationen, ist neben $r$ auch $\ell$ projektiv invariant mit $Q$ verbunden.
\end{remark}

\begin{theorem}[Klassifikation reeller Quadriken]
Zu jeder Quadrik in $P^n$ gibt es eine projektive Transformatio, sodass die abgebildete Quadrik die Darstellung \begin{equation*}
    x_0^2 + x_1^2 + \dots + x_{\ell+1}^2 - x_\ell^2 - \dots - x_{r-1}^2 = 0
\end{equation*}
hat, wobei $\ell , r$ eindeutig sind mit $\ell \geq r-\ell$.
\end{theorem}
\begin{definition}
Die Anzahl der negativen Vorzeichen $s:= r-\ell$ heißt \textit{Index von $Q$}. Das Tripel $(\ell, s, n-\ell-s+1)$ heißt \textit{Signatur von $Q$}
\end{definition}
\begin{example}
Die Kugel hat die Darstellung \begin{equation*}
    -x_0^2 + x_1^2+x_2^2+x_3^2 = 0
\end{equation*}
oder in Normalform
\begin{equation*}
    x_0^2 + x_1^2+x_2^2-x_3^2 = 0
\end{equation*}
mit Signatur $(3,1,0)$ bzw. $(+++-)$.
\end{example}
\begin{example}
Ein Kegel hat die Darstellung\begin{equation*}
    x_1^2+x_2^2-x_3^2 = 0
\end{equation*}
oder in Normalform
\begin{equation*}
    x_0^2 + x_1^2-x_2^2+ 0 = 0
\end{equation*}
mit Signatur $(2,1,1)$ bzw. $(++-0)$.
\end{example}
\begin{remark}[Typen von \textbf{reellen} Quadriken]
\begin{center}
    \begin{tabular}{c|c|c|c|c}
         n&Signatur&Gleichung&Symbol& Bezeichnung  \\ \hline \hline \hline
         1& $(+0)$ &$x_0^2 = 0$&& Doppelpunkt\\\hline
         &$(++)$&x_0^2+x_1^2 = 0 &&/\\
         &$(+-)$&x_0^2-x_1^2 = 0&&Punktepaar\\\hline\hline
         2&$(+00)$& x_0^2=0&& Doppelgerade\\
         &$(++0)$&x_0^2+x_1^2= 0&& Doppelpunkt\\
         &$(+-0)&x_0^2 -x_1^2 = 0&& Geradenpaar \\\hline
         &$(+++)& x_0^2+x_1^2+x_2^2 = 0 && /\\
         &$(++-)& x_0^2+x_1^2-x_2^2 = 0&& regulärer Kegelschnitt\\\hline\hline
         3& $(+000)& x_0^2 = 0&&Doppelebene\\
         &$(++00)&x_0^2+x_1^2 = 0&&Doppelgerade\\
         &$(+-00)& x_0^2-x_1^2 = 0 && Ebenenpaar\\
         &$(+++0)& x_0^2+x_1^2+x_2^2 = 0&& Doppelpunkt\\
         &$(++-0)& x_0^2+x_1^2-x_2^2 = 0&& Kegel\\\hline
         &$(++++)& x_0^2+x_1^2+x_2^2+x_3^2 = 0&& /\\
         &$(+++-)&x_0^2+x_1^2+x_2^2-x_3^2 = 0 && Ovalquadrik\\
         &$(++--)&x_0^2+x_1^2-x_2^2-x_3^2 = 0 && Ringquadrik\\\hline\hline
         n & $(+0\dots0)& x_0 = 0&& Doppelhyperebene\\
         &$(++0\dots0)&x_0^2+x_1^2 = 0 && Doppelhypergerade\\
         &$(+-0\dots0)&x_0^2-x_1^2 = 0 && Hyperebenenpaar\\
         &$(+\dots+0) & x_0^2 + \dots + x_{n-1}^2 = 0 && Doppelpunkt\\
         &$(+\dots+-0)&x_0^2+\dots + x_{n-2}^2-x_{n-1}^2 = 0&& Hyperkegel\\\hline
         &$(+\dots+)& x_0^2+\dots+x_n^2 = 0&\emptyset& /\\
         &$(+\dots+-)&x_0^2+\dots+x_{n-1}^2-x_n^2 = 0 && Ovalquadrik\\
         &$(+\dots+--)$&x_0^2+\dots+x_{n-2}^2-x_{n-1}^2-x_n^2 = 0 && Ringhyperquadrik
    \end{tabular}
\end{center}
\end{remark}



\end{document}