\documentclass[11pt]{article}
\usepackage{ {util/personalmacros} }
\usepackage{tikz-cd}
%\usetikzlibrary{arrows,calc, angles ,patterns}

\title{Lecture Notes} 

\author{Konstantin Mark\\
Projektive Geometrie\\ 
\textsc{TU Wien}
}



\begin{document}
\maketitle

\section{Quadriken}


\begin{definition}
    Sei $A\in \mathbb R^{(n+1)\times(n+1)}\backslash\{0\}$. Die Punktmenge $Q= \{[x]\in P^n|x^tAx = 0$ heißt \textit{Quadrik} oder \textit{Hyperfläche 2. Ordnung}.
\end{definition}

\begin{remark}
    Für $n=2$ ergeben sich Kegelschnitt. Ohne Einschränkung kann $A$ symmetrisch gewählt werden. $\lambda A$ beschreibt den gleichen Kegelschnitt wie $A$, sofern $\lambda \neq 0$. 
\end{remark}

\begin{remark}
    Sei $T\in \mathbb R^{(n+1)\times(n+1)}$ regulär. Dann beschreibt $T$ eine projektive Transformation $\kappa$. $\kappa(Q)$ ist wieder eine Quadrik mit Matrix $T^{-T}AT^{-1}$
\end{remark}

\begin{definition}
    $[b]\in Q$ heißt \textit{regulär} wenn $b^TA\neq 0$, ansonsten \textit{singulär}.
\end{definition}

\begin{definition}
    Eine Quadrik heißt \textit{regulär}, wenn ihre darstellende Matrix regulär ist. Ansonsten heißt $Q$ \textit{singulär} oder \textit{degeneriert}. 
\end{definition}

\begin{remark}
    Die Punkte einer regulären Quadrik sind regulär.
\end{remark}

\begin{example}
    \begin{itemize}
        \item $x^2+y^2-z^2 = 1$, ein einschaliges Drehhyperboloid.!Abbildung 10-1!. Homogenisieren funktioniert wie gewohnt:
            \begin{equation*}
                \left(\frac{x_1}{x_0}\right)^2 + \left(\frac{x_2}{x_0}\right)^2 - \left(\frac{x_3}{x_0}\right)^2 = 1
            \end{equation*}
            beziehungsweise 
            \begin{equation*}
                -x_0^2 + x_1^2 + x_2^2-x_3^2 = 0
            \end{equation*}
            und in Matrixform 
            \begin{equation*}
                x^T\left(
                \begin{array}{cccc}
                     -1  \\
                     &1\\
                     &&1\\
                     &&&-1
                \end{array}
                \right)x.
            \end{equation*}
        \item $x^2+y^2-z^2 = 0$ ein Drehkegel mit Matrixform 
            \begin{equation*}
                x^T\left(
                \begin{array}{cccc}
                     0  \\
                     &1\\
                     &&1\\
                     &&&-1
                \end{array}\right)x.
            \end{equation*} woran man erkennt, dass der Kegel nicht regulär ist. Der Punkt $[(1,0,0,0)^T]$ ist singulär, alle anderen Punkte sind regulär.
    \end{itemize}
\end{example}

\begin{theorem}
    Regularität bzw. Singularität eines Punktes ist invariant unter projektiven Transformationen.
\end{theorem}

\begin{proof}
    Sei $[b]\in Q$ ein regulärer Punkt, also $b^TA\neq 0$ !Abbildung 10-2!. Es gilt 
    \begin{equation*}
        (Tb)^T\cdot(T^{-T}AT^{-1}) = b^TAT^{-1}\neq 0.
    \end{equation*}
\end{proof}

\subsection{Schnitt einer Quadrik mit einer Geraden}

Seien $[a]\neq [b]\ini P^n$. Dann bedeutet $([a]\lor [b])\cap Q$ genau
\begin{align}
   (\lambda a + \mu b)^T\cdot A\cdot (\lambda a+\mu b) = 0&\Leftrightarrow\\
    \lambda^2 a^TAa + \lambda \mu a^TAb + \underbrace{\mu\lambda b^TA\cdot a}_{= a^TA^Tb = a^TAb} + \mu^2 b^TAb = 0&\Leftrightarrow\\
    \lambda^2 a^TA + 2\lambda \mu a^TAb + \mu^2 b^TAb = 0& \label{schnittbedingung}
\end{align}

\begin{example}Betrachte die Abbildung mit Matrix 
    $$\left(\begin{array}{ccc}
         -1\\&1\\&&1 
    \end{array}\right).$$ Der Schnitt dieses Kreises mit Ferngerade ist leer. (keine reellen Schnittpunkte) !Abbildung 10-3!
\end{example}
\begin{example}
    Sei $[a]\neq [b]\in Q, [\lambda a+ \mu b]\in Q$ für $\lambda,\mu \in \mathbb R\backslash\{0\}$. (\ref{schnittbedingung}) impliziert 
    \begin{align*}
        \underbrace{\lambda^2a^TAa}_{=0} +2\lambda\mu a^TAb + \underbrace{\mu^2b^TAb}_{=0} = 0\\
        \Rightarrow  \underbrace{\lambda \mu }_{=0}a^TAb = 0 
    \end{align*}
    also $a^TAb=0$ und (\ref{schnittbedingung}) verschwindet identisch.\\
    Es gilt also $[\lambda a + \mu b]\in Q \forall (\lambda,\mu)\neq (0,0)$ und damit $[a]\lor [b] \in Q$. Liegen also $3$ verschiedene Punkte einer Geraden in $Q$, so liegt die gesamte Gerade darin.
\end{example}

\end{document}