\documentclass[11pt]{article}
\usepackage{ {util/personalmacros} }
\usepackage{tikz-cd}
%\usetikzlibrary{arrows,calc, angles ,patterns}

\title{Lecture Notes} 

\author{Konstantin Mark\\
Projektive Geometrie\\ 
\textsc{TU Wien}
}



\begin{document}
\maketitle

\subsection{Quadriken}


\begin{definition}
Sei $A\in \mathbb R^{(n+1)\times(n+1)}\backslash\{0\}$. Die Punktmenge $Q= \{[x]\in P^n|x^tAx = 0$ heißt \textit{Quadrik} oder \textit{Hyperfläche 2. Ordnung}.
\end{definition}
\begin{remark}
Für $n=2$ ergeben sich Kegelschnitt. Ohne Einschränkung kann $A$ symmetrisch gewählt werden. $\lambda A$ beschreibt den gleichen Kegelschnitt wie $A$, sofern $\lambda \neq 0$. 
\end{remark}
\begin{remark}
Sei $T\in \mathbb R^{(n+1)\times(n+1)}$ regulär. Dann beschreibt $T$ eine projektive Transformation $\kappa$. $\kappa(Q)$ ist wieder eine Quadrik mit Matrix $T^{-T}AT^{-1}$
\end{remark}
\begin{definition}
$[b]\in Q$ heißt \textit{regulär} wenn $b^TA\neq 0$, ansonsten \textit{singulär}.
\end{definition}
\begin{definition}
Eine Quadrik heißt \textit{regulär}, wenn ihre darstellende Matrix regulär ist. Ansonsten heißt $Q$ \textit{singulär} oder \textit{degeneriert}. 
\end{definition}
\begin{remark}
Die Punkte einer regulären Quadrik sind regulär.
\end{remark}
\begin{example}
\begin{itemize}
    \item $x^2+y^2-z^2 = 1$, ein einschaliges Drehhyperboloid.!Abbildung 10-1!. Homogenisieren funktioniert wie gewohnt:
    \begin{equation*}
        \left(\frac{x_1}{x_0}\right)^2 + \left(\frac{x_2}{x_0}\right)^2 - \left(\frac{x_3}{x_0}\right)^2 = 1
    \end{equation*}
    beziehungsweise \begin{equation*}
        -x_0^2 + x_1^2 + x_2^2-x_3^2 = 0
    \end{equation*}
    und in Matrixform 
    \begin{equation*}
        x^T\left(\begin{array}{cccc}
             -1  \\
             &1\\
             &&1\\
             &&&-1
        \end{array}\right)x.
    \end{equation*}
    \item $x^2+y^2-z^2 = 0$ ein Drehkegel mit Matrixform \begin{equation*}
        x^T\left(\begin{array}{cccc}
             0  \\
             &1\\
             &&1\\
             &&&-1
        \end{array}\right)x.
    \end{equation*} woran man erkennt, dass der Kegel nicht regulär ist. Der Punkt $[(1,0,0,0)^T]$ ist singulär, alle anderen Punkte sind regulär.
\end{itemize}
\end{example}


\end{document}