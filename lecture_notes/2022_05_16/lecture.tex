\documentclass[11pt]{article}
\usepackage{ {util/personalmacros} }
\usepackage{tikz-cd}
%\usetikzlibrary{arrows,calc, angles ,patterns}

\title{Lecture Notes} 

\author{Konstantin Mark\\
Projektive Geometrie\\ 
\textsc{TU Wien}
}



\begin{document}
\maketitle

\begin{definition}
Sei $g$ eine Gerade, $Z$ ein Punkt mit $Z\notin g$.. Die Abbildung \begin{equation*}
    p: \left\{\begin{array}{ccc}
         \{\text{Geradenbüschel von } Z\}&\to &g  \\
         x&\mapsto &x\cap g =:X 
    \end{array}\right.
\end{equation*}
heißt Perspektivität.
!Abbildung 9-1!
\end{definition}
\begin{definition}
Analog heißt auch die Abbildung 
\begin{equation*}
    p: \left\{\begin{array}{ccc}
         g&\to &\{\text{Geradenbüschel von } Z\}  \\
         X&\mapsto &X\lor Z =:x 
    \end{array}\right.
\end{equation*} Perspektivität.
\end{definition}
\begin{remark}
!Abbildung 9-2!
\end{remark}
\begin{definition}
Hintereinanderausführungen von Perspektivitäten heißen Projektivitäten.
\end{definition}
\begin{definition}
eine projektive Transformation $\kappa$ einer Fixpunkthyperebene (auch Achse genannt) heißt perspektive Kollineation. Es werden zwei Fälle unterschieden:
\begin{itemize}
    \item Gilt $Z\in H$, so heißt $\kappa$ Elation.
    \item Gilt $Z\not\in H$ so heißt $\kappa$ Homologie
\end{itemize}
\end{definition}
\begin{example}(Analytische Beschreibung der perspektiven Kollineation
\begin{enumerate}
    \item[(Homologien)] Sei $z, u\in \mathbb R^{n+1}\backslash \{0\}, \langle z,u\rangle\neq 0$.. Sei $c\in \mathbb R$. Betrachte die lineare Abbildung
    \begin{equation*}
    f: \left\{
    \begin{array}{ccc}
         \mathbb R^{n+1}& \to &\mathbb R^{n+1} \\
         x&\mapsto &x+(c-1)\frac{\langle x, u\rangle}{\langle z,u\rangle}z
    \end{array}\right.
    \end{equation*} und definiere die dazu zugehörige projektive Abbildung 
    \begin{equation*}
        \kappa: \left\{
        \begin{array}{ccc}
             P^n\backslash P(\ker f)&\to&P^n\\
             {[x]}&\to &\kappa([x]) = [f(x)] 
        \end{array}\right.
    \end{equation*}
    Sei $H:= \{[x]\in P^n|\langle x,u\rangle = 0\}$. Es gilt $\ker f = \{x\in \mathbb R^{n+1}|f(x) = 0\}$. $x\in \ker f$ genau dann wenn $x+(c-1)\frac{\langle x, u\rangle}{\langle z,u\rangle}z$. Es gibt ein $\lambda\in \mathbb R$ mit $x = \lambda z$ und $0 = f(\lambda z) = \lambda c z$
    \begin{itemize}
        \item Fall $c = 0$: Es gilt $\lambda\in \mathbb R$ und damit $f(\lambda z = f(x) = 0$ und $[z] = \ker f$.
        \item Fall $c \neq 0$: $\lambda = 0\Rightarrow x = 0\Rightarrow \ker f = \{0\}$ und $f$ ist bijektiv und $\kappa$ eine projektive Transformation. 
    \end{itemize}
    Sei $Y = [y] \in H$. Dann gilt $\langle y, u\rangle = 0$ und $\f(y) = y+ (c-1)\frac{\langle y, u\rangle}{\langle z, u\rangle}z = y$ und wegen $Y = \kappa(Y)$ ist $H$ eine Fixpunkthyperebene (Achse).\\
    $\kappa(X)\in (Z\lor X)$
    \begin{enumerate}
        \item Fall $c=0$: \begin{equation*}
            \langle f(x), u\rangle = \left\langle x- \frac{\langle x, u\rangle}{\langle z,u\rangle}z, u \right\rangle= \langle x,u\rangle - \frac{\langle x, u\rangle}{\langle z,u\rangle}\langle z,u\rangle = 0
        \end{equation*}
        also $\kappa (X) \in H$. Insbesondere gilt $\kappa\circ \kappa = \kappa$.
        \item Fall $c\neq 0$:
            \begin{equation*}
                f(z) = z+(c-1)\frac{\langle z, u\rangle}{\langle z, u\rangle}
            \end{equation*} und damit $[f(z)] = [z]$. Es ist also $Z$ ein Fixpunkt.
            \end{enumerate}
    Es 
\end{enumerate}
!Abbildung 9-4!
\end{example}
\begin{remark}
Eine perspektivische Kollineation kann involutorisch sein. Eine solche Abbildung heißt projektive Spiegelung. $\kappa|_S$ ist eine Autoprojektivität und sogar Involution: $\mathrm{DV}(X, \kappa(X), Z, F) = -1$. 
\\
!Abbildung 9-5!
\\
Bei einer Affinspiegelung sind Abstand von $X$ und $\kappa(X)$ zu $H$ gleich
\end{remark}
\begin{remark}
Ist in $\mathrm{DV}(X, \kappa(X), Z, U) = -1$ der Punkt $U$ ein Fernpunkt, so ist $M$ der Mittelpunkt von $XY$. Insbesondere ist, falls $Z$ der Fernpunkt der rothogonalen Richtung zu $H$ ist die Spiegelung.
!Abbildung 9-6!
\end{remark}
\begin{remark}
Ist die Achse einer Spiegelung eine Fernhyperebene, so ist die Spiegelung eine Punktspiegelung !Abbildung 9-7!
\end{remark}
\begin{remark}
Nachdem $\mathrm{DV}(X,\kappa(X),Z,F) = -1$ ist, wird $\kappa$ auch harmonische Spiegelung (oder harmonische Involution) genannt.
\end{remark}
\end{document}