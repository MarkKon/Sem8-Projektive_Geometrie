\documentclass[11pt]{article}
\usepackage{ {util/personalmacros} }
\usepackage{tikz-cd}
%\usetikzlibrary{arrows,calc, angles ,patterns}

\title{Lecture Notes} 

\author{Konstantin Mark\\
Projektive Geometrie\\ 
\textsc{TU Wien}
}



\begin{document}
\maketitle
\begin{remark}
\begin{equation}
  Q\dots, x^TAx = 0
\end{equation}
Wir können quadratische Matrizen $A\in \mathbb R^{(n+1)\times(n+1)}$ mit Vektoren im $\mathbb R^{(n+1)\cdot(n+1)}$ identifizieren z.B. $

Basis für den Vektorraum der symmetrischen Matrizen


\end{remark}

\end{document}